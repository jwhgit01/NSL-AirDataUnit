% Define the Document Class
\documentclass[10pt,onecolumn]{article}
%
% Classes include 	{article}, {amsart}, {IEEEtran}, {proc}, {report}, ...
% Options include 	[10pt], [letterpaper], [draft], [fleqn] (left-aligned equations),
%					[leqno] (formula numbering on the left), [titlepage] or [notitlepage],
%					[onecolumn] or [twocolumn], [twoside] or  [oneside], [landscape],

% Possible Useful Packages
\usepackage{amsmath} 	% AMS Math package
\usepackage{amssymb} 	% AMS extended symbols package
\usepackage{cancel}		% \cancelto{<value>}{expression}
\usepackage{amsthm} 	% AMS Theorem Package - Provides extended theorem environments
\usepackage{amsfonts}	% AMS Fonts Package - Provides extended fonts
\usepackage{mathrsfs}	% Raph Smith's Formal Script font. \mathscr command
\usepackage{relsize}
\usepackage{arydshln}	%
%\usepackage{xfrac}
%\usepackage{multirow}	% Create tabular cells spanning multiple rows
%\usepackage{lscape}	% Place selected parts of a document in landscape
\usepackage{parskip} 	% Layout with zero \parindent, non-zero \parskip
\usepackage[left=0.75in, right=0.75in, top=1in, bottom=0.75in]{geometry}	% customize page layout
%\usepackage{lineno} 	% Adds line numbers. \linenumbers command
\usepackage{graphicx} 	% Add figures
\usepackage{caption,subcaption,wrapfig}
\usepackage{float}
\usepackage{hyperref}
\usepackage{fancyhdr}	% Add Headers and Footers
\usepackage{enumitem}
\usepackage[dvipsnames]{xcolor}
%\usepackage[numbered,framed]{matlab-prettifier} % Pretty Matlab formatting
\usepackage{todonotes,soul}

% Shortcut for stretching matrices and arrays
\makeatletter
\renewcommand*\env@matrix[1][\arraystretch]{
	\edef\arraystretch{#1}
	\hskip -\arraycolsep
	\let\@ifnextchar\new@ifnextchar
	\array{*\c@MaxMatrixCols c}}
\makeatother

% Template for Figures
%\begin{figure}[h]\begin{center}
%	\includegrabetacs[width=0.5\textwidth]{filename}
%	\caption{caption}
%	\label{fig_label}
%\end{center}\end{figure}

% template for code listing
%\begin{lstlisting}[style=Matlab-editor,numbers=none,frame=none]
%code here
%\end{lstlisting}
%or
%\lstinputlisting[style=Matlab-editor]{filename}

% Document Options
%\everymath{\displaystyle}	% Set the display style of every equation (i.e. including equations within text paragraphs) to be '\displaystyle'. 
%\linespread{1.3}			% Change the line spacing for the entire document by a factor

% hide section numbers but still count
%\usepackage{titlesec}
%\titleformat{\section}{\normalfont\large\bfseries}{}{0pt}{}

% Bold vectors and matrices
\usepackage{bm}
\newcommand{\vect}[1]{\boldsymbol{#1}}

\newcommand{\tol}[2]{\genfrac{}{}{0pt}{}{+#1}{-#2}}

% Derivatives
\newcommand{\pp}[2]{\ensuremath{\frac{\partial #1}{\partial #2}}}
\newcommand{\dd}[2]{\ensuremath{\frac{\mathrm{d} #1}{\mathrm{d} #2}}}

% Title, Author, Date
\title{Vaned Air Data Unit}
\author{Jeremy W. Hopwood}
\date{\today}

% Begin the Document
\begin{document}
	
% Make the Title
\maketitle

% table of contents
\tableofcontents

\section{Construction}

	\subsection{Bill of Materials}
	
	\subsection{3D Printed Parts}
	
	\subsection{Build Instructions}
	
		\subsubsection{Carbon Fiber Maintube}
			
			The first step is to cut and drill the carbon fiber maintube. The following parts and tools are required for this step:
			\begin{enumerate}
				\item 5/8" x 0.695" carbon fiber tubing
				\item Maintube hole jig (\texttt{hole\_jig\_top.stl} and \texttt{hole\_jig\_bottom.stl})
				\item (6x) 1/4"-20 x 1.5" countersunk screws \& nuts
				\item 1/2", 3/8", and 1/8" carbine drill bits
				\item \#46 drill bit (or next largest fractional size)
				\item Proper equipment and PPE for cutting carbon fiber
			\end{enumerate}
			
			Ensure one end of the carbon fiber tubing is flat and cut parallel to the tube. Loosely assemble the maintube hole jig such that the 1/2~inch and 3/8~inch are concentric. Fit the jig over the carbon fiber tubing such that the end marked ``A'' is flush with the end of the tube. Evenly tighten the 1/4~inch bolts just tight enough to prevent slipping of the carbon fiber tube. 
			
			TODO...
			
			Finally, cut the main tube to the desired length using the proper method and personal protective equipment. For example. see \url{https://www.clearwatercomposites.com/resources/how-to-guides/cut-carbon-fiber-tubes/}. Typically, this is between 2-3 feet. Any longer and the loss of accuracy due to the length of the pitot-static system tubing becomes significant.
			
		\subsubsection{Carbon Fiber Vane Shaft Sleeves}
			
			The next step is to cut the carbon fiber vane shaft sleeves to length. The following parts and tools are required for this step:
			\begin{enumerate}
				\item 3/8" X 1/2" carbon fiber tubing
				\item Proper equipment and PPE for cutting carbon fiber
			\end{enumerate}
			
			Cut two sections of carbon fiber tube to a length of $1.75\pm0.01$~inches using the proper method and personal protective equipment. Ensure the ends are clean and square.
		
		\subsubsection{Vane Shafts and Couplings}
		
			The next step is to cut and machine the vane shafts as well as the couplings that connect the shaft to the encoder. The following parts and tools are required for this step:
			\begin{enumerate}
				\item 1/8" 304 stainless steel rod
				\item (2x) 1/8" to 1/8" bore rigid aluminum set screw couplings
				\item Fine band saw (or rotary tool with cut-off wheel)
				\item Bench-top lathe (or drill press, metal file, \& sandpaper)
				\item Bench-top mill (or rotary tool/metal files)
			\end{enumerate}
			Note the alternate tools given by no means enable the ideal machining method, but will work.
			
			First, cut two stainless steel shafts each to 2 inches in length. This ideally should be done with a find band saw, but a rotary tool with a cut-off wheel also works well. Be sure the ends are square.
			
			Next, machine on the the ends of the shaft down to a flat surface. This flat surface should extend 0.3$\tol{0.01}{0.00}$~inches down the shaft and 0.03$\pm$0.003~inches down towards the shaft center line. A bench-top mill works best for this, but careful work with a rotary tool finished off with a fine metal file works too.
			
			Next, the outer diameter of the aluminum shaft couplings needs to be reduced. One way to do this is to install the coupling onto any left over raw material from the shaft using the included set screws. Using a lathe, machine down the outer diameter of the coupling (including the set screws) to $0.275\pm0.005$~inches. If the hex socket of the set screw would be too shallow after this process, sand/file down the point end of the set screw enough such that after the machining process, there is enough hex socket left to torque down the set screw.
		
		\subsubsection{Pitot-Static Tubing and Kiel Probe}
		
			The next step is to cut and install the pitot-static tubing and assemble the kiel probe. The following parts and tools are required for this step:
			\begin{enumerate}
				\item Kiel probe (\texttt{kiel\_probe.stl}) 
				\item 3003 aluminum tube, 0.014" Wall Thickness, 3/32" OD
				\item PVC soft plastic tubing, 3/32" ID, 5/32" OD
				\item Fine band saw (or rotary tool with cut-off wheel)
				\item 5-minute epoxy
			\end{enumerate}
		
			Cut the aluminum tube into a 1~inch section and a 5~inch section, being sure to keep the ends cleanly cut. Insert the 5~inch tube into the center hole of the 3D printed kiel probe until 0.5~inches protrude from the aft end. This is the dynamic pressure port. Then, insert the 1~inch tube into the offset hole of the aft end of the kiel probe, which connects to the static pressure ports. Using the 5-minute epoxy, adhere and seal the two aluminum tubes to the 3D printed kiel probe. 
			
			Cut two lengths of the PVC soft plastic tubing to the appropriate length. These should be about 6 inches longer than the length of the carbon fiber maintube. After the epoxy has fully cured, slide the PVC tubing over the aluminum tubes.
		
		\subsubsection{Assemble Air Data Unit Boom Assembly}
		
			The next step is to assemble the air data unit boom assembly. \emph{This step is irreversible and involves permanent assembly.} The following parts and tools are required for this step:
			\begin{enumerate}
				\item Pitot-static tubing and kiel probe assembly
				\item Machined carbon fiber maintube 
				\item Machined carbon fiber vane shaft sleeves
				\item Machined aluminum shaft couplings
				\item Miniature rotary encoders and included nuts
				\item Red thread locker
				\item 5-minute epoxy
				\item Oscilloscope and 3.3V-5V power supply
			\end{enumerate}
		
			First, it is important to ensure the rotary encoder zero angle is \emph{not} aligned with the direction pointing out the front of the air data unit. This is because the smallest pulse width modulation (PWM) value of 1~$\mu$s is not accurately measured by the microcontroller. For this reason, the 180-degree shaft position of the rotary encoders should be facing out the front of the air data unit. To check this, first connect the $\mathrm{V_{in}}$ and GND pinouts of the rotary encoder to a 3.3V-5V power supply. Then, connect the oscilloscope data pin to the data pinout of the encoder and the oscilloscope ground to the same ground as the encoder/power supply. Turn the encoder to its position such that the pulse width is approximately at half its maximum value. Lightly mark the shaft of the encoder and the encoder body to make note of this position. 
			
		\subsubsection{Wire and Solder Microcontroller}
		
			\paragraph{Required Parts and Materials}
			
		\subsubsection{Program Microcontroller}
		
			\paragraph{Required Parts and Materials}
	
	\subsection{PX4 Integration}

\section{User Guide}

	\subsection{Mounting Considerations}
	
	\subsection{Data Analysis}
	
	\subsection{Research Efforts}

\appendix
\section{Source Code}

	\subsection{Microcontroller code}
	
	\subsection{PX4 code}

\section{Engineering Drawings}

	\subsection{Kiel Probe}
	
	\subsection{Vane}
	
	\subsection{Calibration Fixture}
	
	\subsection{Hose Guide}
	
\end{document}